\subsection{Binary Release}
\label{sec:install_bin}

Currently, a binary release is provided for Linux Ubuntu 12.04LT.
It is provided as a .tar.gz file. The user only needs to untar it
and install it in the desired folder, name it \texttt{KISTA\_PATH}.

After uncompressing it, the user will see the following libraries:

\begin{itemize}
\item \textbf{/include} : sources with the kista.hpp header and other header files, for development of KisTA models with its C++ API.
\item \textbf{/lib} : binary libraries with the static (\texttt{libkista.a}) and dynamic (\texttt{libkista.so}) version of the library. It also contains the lib\texttt{systemlevel\_bypass.a}, to enable a standalone SystemC compilation of the model.
\item \textbf{/bin} : folder with binary version of the tool, the \texttt{kista-xml} executable. It also provides an executable for checking that the XML Kista model abides the KisTA model schema.
\item \textbf{/examples} : folder with KisTA examples for both the C++ and for the XML interface
\end{itemize}

An easy installation producedure, if you only are using the XML front-end is:
\begin{itemize}
\item Add \texttt{KISTA\_PATH/bin} to the \texttt{PATH} environment variable
\item Add \texttt{KISTA\_PATH/lib} to the \texttt{LD\_LIBRARYPATH} environment variable
\end{itemize}

If you are using the C++ API, so you use KisTA as a SystemC/C++ library, then you should consider as well:

\begin{itemize}
\item Add \texttt{KISTA\_PATH/include} to your include compilation switch
\item Add the switch \texttt{-DSC\_INCLUDE\_DYNAMIC\_PROCESSES} at compilation time
\end{itemize}

\subsection{Sources}
\label{sec:install_bin}

KisTA has become an open development and its sources are available in the following
GIT repository:\\ 
\\
\url{http://github.com/nandohca/kista}\\
\\
The commands to download the sources in a path of your system called \texttt{SOFT\_PATH} are
the following ones

\begin{verbatim}
$cd SOFT\_PATH
$git clone http://github.com/nandohca/kista kista
\end{verbatim}

This creates the directory \texttt{kista} within \texttt{SOFT\_PATH}, such
then \texttt{KISTA\_PATH=SOFT\_PATH\\kista}.
Then, move to the root of the KisTA sources, and compile the library.
Use the following commands for it:

\begin{verbatim}
$cd KISTA\_PATH
$make
$make install
\end{verbatim}

This compiles and install the KisTA library, enabling its use as a SystemC-based
C++ library (C++ API).
The KisTA \texttt{\\include} and \texttt{\\lib} directories are generated within
the \texttt{KISTA\_PATH} directory, with the headers and object libraries.
%
All the examples but the ones using the XML interface can be run with this.

The KisTA Makefile contains also the \texttt{clean} rule,
to eliminate the object files generated during the compilation,
and the \texttt{ultraclean} rule, to elimitate as well the created library.

Once the KisTA library has been compiled and installed it is
possible to compile and install the KisTA XML front-end.
For it, the following commands shall be used, asuming you are in
the KISTA\_PATH:

\begin{verbatim}
$make kista-xml
$make kista-xml install
\end{verbatim}

This will create a \texttt{\\bin} folder where an executable file, called \texttt{kista-xml}
will be created and copied.
The \texttt{kista-xml} is the one capable to parse an read a KisTA description in XML,
and launch the static analysis and the simulations.



