
\subsection{Scheduling Time Penalties}
\label{sec:scheduling_time_penalties}

The user can enable or disable scheduling time penalties.
%
The following functions are available for enabling/disabling scheduling time penalties:

%\begin{table}[t]
\begin{lstlisting}[style=KistaCodeStyle,caption={API for enabling/disabling scheduling time penalties.},label=fig::sched_time_penal]
void enable_scheduling_times();  
void disable_scheduling_times();
\end{lstlisting}
%\end{table}
These functions have to be called at elaboration time, i.e. before \texttt{sc\_start}.

By default, current release diables scheduling time penalties.
%
However, in order to ensure that your KisTA model is release independent, the user can explicitly disable them. 

When the user enables scheduling times, the default constant scheduling penalty is applied
every time a task scheduling is performed by an scheduler instance.
%
The value stated for the variable \texttt{DEFAULT\_SCHED\_TIME\_NS} in \texttt{defaults.hpp} is used for
every scheduler instance of the model.

It is possible to configure the constant scheduling penalties through the \texttt{set\_scheduling\_time} function
(to be called at elaboration time).
%
This function is overloaded, as shown in Listing~\ref{fig::set_sched_time_penal}.
If only the scheduling time is passes as an argument, a global scheduling time is settled (overriding the default scheduling time)
which is considered for all the scheduler instances of any type, but any instance which an instance specific scheduling time has
been configured for.
%
In case, a scheduler instance is passed as a second argument, then such scheduling time will be applied only for such an 
scheduling instance.

\begin{lstlisting}[style=KistaCodeStyle,caption={API for setting user specific constant scheduling times.},label=fig::set_sched_time_penal]
void set_scheduling_time(sc_time time, scheduler *sched);
void set_scheduling_time(sc_time time);
\end{lstlisting}

In order to enable a more accurate model of scheduling penalties,
KisTA enables the use of \emph{scheduling time calculators}.
%
A scheduling time calculator is a function in charge of calculating
the scheduling time of a given scheduler instance, given its current
state.

KisTA enables the association of a default set of scheduling time calculators.
Listing~\ref{fig::set_sched_time_default_cal} shows the function use to state such an association.
\begin{lstlisting}[style=KistaCodeStyle,caption={API for setting user specific constant scheduling times.},label=fig::set_sched_time_default_cal]
void set_default_scheduling_time_calculators();
\end{lstlisting}
Specifically, what this function does is to associate a scheduling time penalty proportional to the number of tasks ready
for execution for all scheduler instances but for scheduler instance with a static scheduling policy.
For the latter, the global constant scheduling time configured is applied.

The user might want to have a more detailed and scheduler policy specific scheduling time penalty model.
KisTA allows the association of user-provided  scheduling time calculators.
This is a specific KisTA extension mechanism, explained in Section~\ref{sec:ext_sched_time}.

