\subsubsection{Using the C++ API}
\label{sec:app_initialization_cpp_api}

KisTA provides an API, shown in Listing~\ref{list:fun_val_api} for enabling and configuring a functional validation facility.
%
It is useful to collect any functional error detected across the model.
By reporting such error through the KisTA functional validation facility, the user delegates on this
facility the later report of the error.
Moreover, the facility can be configured to allow a certain number of \emph{failures}, such only after a given
threshold an error is actually reported. 
This facilitates to perform validations where some marging to le pass a bounded amoung of errors due to rounding, etc, is possible.
%
%\begin{table}[t]
\begin{lstlisting}[style=KistaCodeStyle,caption={API for enabling and configuring functional validation facility.},label=list:fun_val_api]
void enable_functional_validation();

bool functional_validation_enabled();

void set_functional_failure_threshold(unsigned int);

void set_functional_failure();

unsigned int get_functional_failures();
\end{lstlisting}
%\end{table}

The validation facility can be created by means of the call \texttt{enable\_functional\_validation()} function.
The \texttt{enable\_functional\_validation()} function has to be called before the simulation start.
It creates a \emph{functionality validator}.

Once it has been done, then, the KisTA model can call the \texttt{set\_functional\_failure()} function from any system or environment task
during the simulation, e.g. associated to the detection of a functional issue.
The code detecting such issue is encoded by, and so responsability of, the user.
A single call to \texttt{set\_functional\_failure()} will set the functionality validator into a failure state,

At the end of the simulation the functional validator  will report an error, indicating the issue, if, and only if,
the set of functional failures notified is bigger or equal to a threshold.
Such a threshold is 1 by default.
The function \texttt{set\_functional\_failure\_threshold} enables to change the threshold required to raise the error.

The KisTA-XML user can call \texttt{set\_functional\_failure()} from within the task functional code (no matter system or environment), as long as such code is executed during the simulation phase.

\subsubsection{Using the XML front-end}
\label{sec:app_initialization_xml}

When the XML front-end is used, the user can enable the functional validation from the XML KisTA description.
For it, the \texttt{functional\_validaton} entry is set in the configuration section,
as illustrated in Listing~\ref{list:xml_enable_functional_validation},
which is equivalent to calling the \texttt{enable\_functional\_validation} call.

%\begin{table}[t]
\begin{lstlisting}[language=XML, caption={Enabling functional validation from the XML description.}, label=list:xml_enable_functional_validation]
<kista_configuration>
        ...
        <functional_validation value="true"/>
        ...
</kista_configuration>
\end{lstlisting}
%\end{table}

%
Notice that if the XML front-end if used, the functionality attached will still use the C++ API to notify the failure events (with \texttt{set\_functional\_failure})
from within the task code, or by calling \texttt{get\_functional\_failures} for report from the functional hook for wrap up actions.
%
Currently, there is no XML entry to set the threshold of functional failures for raising an error.
A workaround is to employ an environment the initialization function (see Section~\ref{sec:app_initialization}) to call \texttt{set\_functional\_failure\_threshold} function.

If functional the validation option is not settled, then the default value is false and the functionality validator is not created by the KisTA-XML front-end.
This does not requires the user to remove calls to the \texttt{set\_functional\_failure()} function within the tasks code. They are simply ignored and the validations makes no effect.

If the validation is activated, then the error message is reported also in the XML output metrics file.
It is reported with non-fatal severity to, eventually enable tools like Multicube Explorer or MOST to continue an exploration and let later detect which solutions did not pass the functional validation.