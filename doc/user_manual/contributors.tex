\textbf{Fernando Herrera} (KTH) has developed the first version of KisTA.

Some features have been added in order to support models with static scheduling and static communication scheduling. For that, the feedback of \textbf{Kathrin Rosvall} (KTH) has been very useful. Discussions with her going deep in details of the model have enabled to add KisTA a bit-accurate WCCT model of the TDMA bus. Moreover, only after such discussion was possible to realise the feasibility to have trade-offs in accuracy vs simulation speed for the modelling of worst-case communication penalties of TDMA communications.

Many flaws have been found after the work of integrating KisTA in a joint-analytical and simulation-based Design Space Exploration (DSE) framework, where \textbf{Edoardo Paone} (PoliMi) has been a first actor. Moreover, Edoardo has also been co-authoring a KisTA model of the Voice Activty Detection example, used to show the afore-mentioned JAS-DSE flow.

A.Proff. \textbf{Ingo Sander} (KTH) has supported the integration of KisTA framework with other KTH/ES tools, and in the definition of the features that, not only have been already used, but which will be eventually used according the avolution of other tools, such as analytical-DSE. He also helpt to realise the distinctive features of KisTA, so to cooperate in the publication of this framework. 

Last, but not least, some features of this framework are based on the ones supported by the SCoPE tool, whose main father is \textbf{Hector Posadas}(UC). He explained me lot of things relating SCoPE features. I tried to reproduce some of them, but without being tied to the implementation details, or dependencies (eg. KisTA XML interface is based on libxml2, instead of in xpat)  or to some features of SCoPE (e.g., KisTA targets facilitate predictable communication infrastructures, beyond TLM interfaces can be used or not). 
However, I think that, without the kind character and discussions hold with Hector, and the valuable knowledge from him, I doubt i could have tackled this work for my research in KTH in time.


