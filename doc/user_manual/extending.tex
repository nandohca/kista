
KisTA is a tool specifically suitable for its own extension.
KisTA is C++ and SystemC-based, which facilitates the extension from the sources.
KisTA code style (variable naming, comments, etc) has aimed an easier code understanding.

Moreover, KisTA provides specific facilities for the extension of the following features:

\begin{itemize}
\item Scheduling Policies
\item Penalty Model: Scheduling Times
\item Static Analysis
\item Communication Model and related penalties
\end{itemize}

\subsection{Extending the Scheduling Policies}
\label{sec:ext_sched_pol}
TBC

\subsection{Extending the Scheduling Time Calculation}
\label{sec:ext_sched_time}

As mentioned in Section~\ref{sec:scheduling_time_penalties}, the user can provide 
specific scheduling time calculators and associate them to all scheduler policies,
or to specific ones.

KisTA ensures the following priority in the application of a scheduling time calculator
for a specific scheduler instance:
\begin{enumerate}
\item A \emph{scheduler instance specific} scheduling time calculator, or 
\item a \emph{policy specific} scheduling time calculator, applied to every scheduler instance configured under such policy, or 
\item a \emph{generic} scheduling time calculator applicable to every scheduler instances for which either a policy specific or an instance specific scheduling time calculator is not available, or 
\item a scheduler instance specific constant scheduling time (set with \texttt{set\_scheduling\_time(sc\_time, scheduler*)}), or 
\item a global constant scheduling time stated by the user (set with \texttt{set\_scheduling\_time(sc\_time)}), or 
\item a global constant scheduling time stated from the default scheduling time (set with \texttt{DEFAULT\_SCHED\_TIME\_NS} in \texttt{defaults.hpp}).
\end{enumerate}

The latter three ways to state scheduling times where already introduced in Section~\ref{sec:scheduling_time_penalties}.
Here, the first three, which provide an extensible and flexible mechanism to integrate scheduling time penalty models in KisTA
are explained.

A scheduling time calculator function has to have the following interface:

\begin{lstlisting}[style=KistaCodeStyle,caption={Prototype of a scheduling time calculator function.},label=list:sched_time_calculator_if]
typedef sc_time (* SCHTIME_FPTR )(scheduler *sch_ptr);
\end{lstlisting}

The scheduling time calculator function takes as a parameter a pointer to the scheduler instance,
which provides access to state of the scheduler, and also to the task set assigned and all its related information.

KisTA provides the \texttt{set\_scheduling\_time\_calculator} for adding a user provided scheduler calculation function.
Actually, as shown in Listing~\ref{list:add_sched_time_calculators} it is an overloaded function.

\begin{lstlisting}[style=KistaCodeStyle,caption={API to provide scheduling time calculators.},label=list:add_sched_time_calculators]
void set_scheduling_time_calculator(SCHTIME_FPTR fun, scheduler *sched);
void set_scheduling_time_calculator(SCHTIME_FPTR fun, scheduler_t *sched_t);
void set_scheduling_time_calculator(SCHTIME_FPTR fun);
\end{lstlisting}

The three versions require a function pointer to the scheduling time calculator function \texttt{SCHTIME\_FPTR}.
The former version is used for stating instance specific scheduling time calculators.
The a pointer to the scheduler instance (type \texttt{scheduler *}) is passed as a second parameter.
%
The second version is used for stating a policy specific scheduling time calculator
as a second parameter of \texttt{scheduler\_t} type, which describes the type of scheduler
the scheduling time calculator function has to be associated to.
%
The latter version is used for stating a generic scheduling time calculator and does not require any further parameter.

Listing~\ref{list:sched_time_proportional_to_ready_tasks} shows a KisTA provided scheduling time calculator
which calculates the scheduling time proportional to the amount of active and ready to execute tasks
(this calculator is invoked from the \texttt{set\_default\_scheduling\_time\_calculators}, introduced in Section~\ref{sec:scheduling_time_penalties}.

\begin{lstlisting}[style=KistaCodeStyle,caption={API to provide scheduling time calculators.},label=list:sched_time_proportional_to_ready_tasks]
  sc_time scheduling_time;
  unsigned int n_active_tasks;
  taskset_by_name_t *assigned_tasks;
  assigned_tasks = sched->gets_tasks_assigned();
  n_active_tasks = 0;
  for(auto it=assigned_tasks->begin();it!=assigned_tasks->end();it++) {
    if( (it->second->active_signal.read()==true) || 
        (it->second->state_signal.read()=='R') ||
        (it->second->state_signal.read()=='E') )
  	   n_active_tasks++;
	}
	scheduling_time = global_scheduling_time * n_active_tasks;
	return scheduling_time;
\end{lstlisting}

\subsection{Extending the Static Analysis}
\label{sec:ext_stat_anal}
TBC

\subsection{Extending the Communication Infrastructure}
\label{sec:ext_comm}
TBC
