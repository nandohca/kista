KisTA enables the report of \textbf{different types of time metrics} associated both to the application and to the 
the platfom resources.
Kista enables the reporting of metrics offline (at the end of the simulation) and 
during the simulation.
In addition, KisTA-XML front end enables to report metrics \textbf{in a XML file, supporting
the \emph{Multicube}} interface for DSE.

In its latest version, a capability for reporting basic \textbf{energy and power} metrics has been also added.

\subsubsection{Wors-Case and Actual Case measurements}
An important consideration is that KisTA is a tool oriented to the assessment of predictable systems.
Because of that, the tool, bu default, employs worst-case annotations both for computational loads
and for communication loads by default.
%
This means that the assessemnts obtained refer to worst-case situations.
%
In addition, KisTA can provide assessments for annotations based on the actual computational and communicaiton
load, and the time-dependent state of the system.
This enable more accurate assessemnts, either related to specific scenario or to the average case
(this is exploited for instance in the joint-analytical and simulation based DSE methodology proposed
by KTH, PoliMi and UC partners in the FP7 CONTREX project).

\subsubsection{Estimating Actual Computation times}
The user can configure the KisTA tool to consider \emph{actual computational} loads
(instead the worst-case computational loads captured in the model via the WCET and/or WCEI annotations).
%
These actual times have to be provided by an annotation mechanism integrated in the functional code
employed by the KisTA model.
%
Then, for estimating actual computation times, the user needs to do two things:

\begin{itemize}
\item Compile the functional code to add the instrumentation
\item Redefine the consumption macros, e.g. $CONSUME\_T$, in order to employ
the time estimations from the instrumented code, instead from the WCEI/WCET annotations.
\end{itemize}

Currently, KisTA supports one annotation mechanisms, which serves as reference for introducing
new annotations mechanisms in the library.
%
This annotation mechanism relies on the use of the \emph{scope-g++} compilation tool,
provided by the SCoPE tool developed by the University of Cantabria (http://www.teisa.unican.es/scope).
%
SCoPE provide several annotation mechanism.
Among them, the simples one, based on the annotation of the source code
based on the source-levle operations found (opcost).
This method is open and freely available, and therefore, it has been introduced in KisTA
as a reference (without preventing that other more accurate methods, e.g. based on ASM analysis
can be included).

KisTA integrates a pre-compiled version of the SCOPE opcost compiler, which is
installed in $\$KISTA\_HOME\\bin$ (executables \emph{opcost} and \emph{scope-g++}).

This way, the compilation of the functional code can be done with \emph{scope-g++} instead with gcc.
The user can find an example of how the functional code of the VAD is compiled in
{\tt \$KISTA\_HOME/examples/VAD/fun/Makefile}, i.e. the \emph{Makefile} , used to compile
the VAD functionality of the VAD example distributed in KisTA.

The re-definiton of the $CONSUME\_T$ macros is done by including the headers of the annotation plugin.
The simplest way to do so is to add the $-D\_USE\_AUTOMATIC\_CODE\_ANNOTATION$ flag in the 
compilation command.
The user finds an example in the
{\tt \$KISTA\_HOME/examples/VAD/fun/Makefile}
file of the
VAD example (by simply uncommenting the line and re-compiling, the annotation is used).
%
Another alternative is to define the variable by means of a $\#define$ clause in the 
functional code, right before the include of the kista header.

The definition of the $\_USE\_AUTOMATIC\_CODE\_ANNOTATION$ includes the 
$annotation\_plugin/annotation\_plugin.hpp$ generic header. 
This header in turn, selects one among the available annotation headers.
In this case, right the SCoPE annotation header:
$annotation\_plugin/scope\_opcost.hpp$.
%
The SCoPE annotation header provide implementations for the time consumption
macros, e.g. $CONSUME\_T$, which invoke code which computes from the instrumented
code, the actual computational weight as a function of the instructions accounted from the instrumentation,
and as a function of the configured frequency and cycles per instructions associated to the processing element
where the task is running.

\subsubsection{Time related reports}
TBC

\subsubsection{Energy and Power related output metrics}
TBC



