In KisTA, a task is described as a function plus an addociated \emph{task information}.

It is reflected in the following code excerpt:

Listing~\ref{list:kista_task} shows a code excerpt capturing a task named \"task1\".

%\begin{table}[t]
\begin{lstlisting}[style=KistaCodeStyle,caption={Declaration and description of a task in KisTA under it C++ API front-emd.},label=list:kista_task]
 // finite task
 void task1()
 {			
  cout << "task 1 " << " t_begin=" << sc_time_stamp() << endl;
  consume(sc_time(100,SC_MS));	
 }
     
 void starving_task()
 {
   while(true) {		
	cout << "task 1 " << " t_begin=" << sc_time_stamp() << endl;
	consume(sc_time(100,SC_MS));
    yield();
   }		
 }  

 // infinite task, can be preempted during consumption
 void task2()
 {
    while(true) {		
	  cout << "task 2 " << " t_begin=" << sc_time_stamp() << endl;
	  consume(sc_time(100,SC_MS));
	}		
 } 
     
 // infinite task, executes in 0 time (inifinitely fast),
 // but grants control to the scheduler at each iteration
 // and let the scheduler grant access to other tasks
 void task3()
 {
    while(true) {		
	  cout << "task 3 " << " t_begin=" << sc_time_stamp() << endl;
      yield();
	}		
 }            

int sc_main (int, char *[]) {
   
   // declaration and instantiation of 3 tasks
   task_info_t task_info_t1("task1",task1);
   task_info_t task_info_t2("task2",task2);
   task_info_t task_info_t1("task3",task3);

   ...
    
}

\end{lstlisting}
%\end{table}

The KisTA \texttt{task\_info\_t} class enables the declaration of a task.
The constructor is passed the task name as a first parameter and 
a pointer to the function (of type \emph{void ()} ) containing the
behaviour of the task.

Listing~\ref{list:kista_task} example, illustrates a possible 
skeleton describing the task.

If the internal functionality contains an infinite loop, the task
will never yield. 
This will lead to an starvation problem in the simulation, as once
the tasks gets the simulator it will never return the control to other tasks
(even an scheduler modelled in KisTA).

However, KisTA provides several constructs which enable to model several
circunstamces where a real application either yields the control or
looses it.

The \texttt{yield()} function enables to model a yield of control from the
own tasks.
This type of statement is typical in thread libraries.
A yield will be also usually invoked by higher level blocking synchronization
services provided by operative systems.

The \texttt{consume(sc\_time)} function enables to model a delay associated
to a piece of computation within the task. 
A main difference with respect invoking the SystemC \texttt{wait(sc\_time)}
statement is that the consume call will notify the associated KisTA scheduler 
object, such it enables to model the contention of several tasks executing
over the same computational resource, i.e. a processing element.

Therefore, the task executing a \texttt{consume($t_{\delta}$)} statement at 
SystemC time $t_1$ can expect that after the completion of the 
\texttt{consume($t_{\delta}$)} statement, the time  $t_2$ fulfils
 $t_2 \ge t_1$+$t_{\delta}$.
Notice that in the case of executing a \texttt{wait($t_{\delta}$)},
$t_2$ = $t_1$+$t_{\delta}$.


 
